\section{Konec dela}

Na konec dela sodita angleško-slovenski slovarček strokovnih izrazov in seznam
uporabljene literature, morebitne priloge (programska koda, daljša ponovitev
dela snovi, ki je bil obravnavan med študijem \dots) pa neposredno pred ti
enoti. Slovar naj vsebuje vse pojme, ki ste jih spoznali ob pripravi dela, pa
tudi že znane pojme, ki ste jih spoznali pri izbirnih predmetih. Najprej
navedite angleški pojem (ti naj bodo urejeni po abecedi) in potem ustrezni
slovenski prevod; zaželeno je, da temu sledi tudi opis pojma, lahko komentar
ali pojasnilo. Slovarska gesla navajajte z ukazom \verb|\geslo{}{}|, npr.\
\verb|\geslo{continuous}{zvezen}|.

Pri navajanju literature si pomagajte s spodnjimi primeri; najprej je opisano
pravilo za vsak tip vira, nato so podani primeri. Člen literature napišete z
ukazom \verb|\bibitem{oznaka} podatki o viru|, kjer mora \emph{ozmaka} enolično
določati vir.  Posebej opozarjam, da spletni viri uporabljajo paket url, ki je
vključen v~.cls datoteki. Polje ``ogled'' pri spletnih virih je obvezno; če je
kak podatek neznan, ustrezno ``polje'' seveda izpustimo. Literaturo je potrebno
urediti po abecednem vrstnem redu; najprej navedemo vse vire z znanimi avtorji
(tiskane in spletne) po abecednem redu avtorjev (po priimkih, nato imenih),
nato pa spletne vire brez avtorjev, urejene po naslovih strani. Če isti vir
navajamo v dveh oblikah, kot tiskani in spletni vir, najprej navedemo tiskani
vir, nato pa še podatek o tem, kje je dostopen v elektronski obliki.
